%% Generated by Sphinx.
\def\sphinxdocclass{jupyterBook}
\documentclass[letterpaper,10pt,english]{jupyterBook}
\ifdefined\pdfpxdimen
   \let\sphinxpxdimen\pdfpxdimen\else\newdimen\sphinxpxdimen
\fi \sphinxpxdimen=.75bp\relax
\ifdefined\pdfimageresolution
    \pdfimageresolution= \numexpr \dimexpr1in\relax/\sphinxpxdimen\relax
\fi
%% let collapsible pdf bookmarks panel have high depth per default
\PassOptionsToPackage{bookmarksdepth=5}{hyperref}
%% turn off hyperref patch of \index as sphinx.xdy xindy module takes care of
%% suitable \hyperpage mark-up, working around hyperref-xindy incompatibility
\PassOptionsToPackage{hyperindex=false}{hyperref}
%% memoir class requires extra handling
\makeatletter\@ifclassloaded{memoir}
{\ifdefined\memhyperindexfalse\memhyperindexfalse\fi}{}\makeatother

\PassOptionsToPackage{warn}{textcomp}

\catcode`^^^^00a0\active\protected\def^^^^00a0{\leavevmode\nobreak\ }
\usepackage{cmap}
\usepackage{fontspec}
\defaultfontfeatures[\rmfamily,\sffamily,\ttfamily]{}
\usepackage{amsmath,amssymb,amstext}
\usepackage{polyglossia}
\setmainlanguage{english}



\setmainfont{FreeSerif}[
  Extension      = .otf,
  UprightFont    = *,
  ItalicFont     = *Italic,
  BoldFont       = *Bold,
  BoldItalicFont = *BoldItalic
]
\setsansfont{FreeSans}[
  Extension      = .otf,
  UprightFont    = *,
  ItalicFont     = *Oblique,
  BoldFont       = *Bold,
  BoldItalicFont = *BoldOblique,
]
\setmonofont{FreeMono}[
  Extension      = .otf,
  UprightFont    = *,
  ItalicFont     = *Oblique,
  BoldFont       = *Bold,
  BoldItalicFont = *BoldOblique,
]



\usepackage[Bjarne]{fncychap}
\usepackage[,numfigreset=1,mathnumfig]{sphinx}

\fvset{fontsize=\small}
\usepackage{geometry}


% Include hyperref last.
\usepackage{hyperref}
% Fix anchor placement for figures with captions.
\usepackage{hypcap}% it must be loaded after hyperref.
% Set up styles of URL: it should be placed after hyperref.
\urlstyle{same}

\addto\captionsenglish{\renewcommand{\contentsname}{Multivariable Calculus}}

\usepackage{sphinxmessages}



        % Start of preamble defined in sphinx-jupyterbook-latex %
         \usepackage[Latin,Greek]{ucharclasses}
        \usepackage{unicode-math}
        % fixing title of the toc
        \addto\captionsenglish{\renewcommand{\contentsname}{Contents}}
        \hypersetup{
            pdfencoding=auto,
            psdextra
        }
        % End of preamble defined in sphinx-jupyterbook-latex %
        

\title{basics - math}
\date{Jan 07, 2025}
\release{}
\author{basics}
\newcommand{\sphinxlogo}{\vbox{}}
\renewcommand{\releasename}{}
\makeindex
\begin{document}

\pagestyle{empty}
\sphinxmaketitle
\pagestyle{plain}
\sphinxtableofcontents
\pagestyle{normal}
\phantomsection\label{\detokenize{intro::doc}}


\sphinxAtStartPar
\sphinxstylestrong{Argomenti.}
\begin{itemize}
\item {} 
\sphinxAtStartPar
\sphinxstylestrong{Calcolo}
\begin{itemize}
\item {} 
\sphinxAtStartPar
Calcolo multivariabile e calcolo vettoriale in spazi euclidei 2D e 3D

\item {} 
\sphinxAtStartPar
Algebra lineare e multilineare su spazi con prodotto interno

\item {} 
\sphinxAtStartPar
Calcolo lineare e multilineare su spazi con prodotto interno

\end{itemize}

\item {} 
\sphinxAtStartPar
\sphinxstylestrong{Geometria}
\begin{itemize}
\item {} 
\sphinxAtStartPar
Geometria differenziale

\end{itemize}

\item {} 
\sphinxAtStartPar
\sphinxstylestrong{Calcolo delle variazioni} (qui e/o nella scuola superiore?)

\item {} 
\sphinxAtStartPar
\sphinxstylestrong{Calcolo complesso}
\begin{itemize}
\item {} 
\sphinxAtStartPar
Analisi complessa

\item {} 
\sphinxAtStartPar
\sphinxstylestrong{Teoria delle trasformate}: Fourier, Laplace

\end{itemize}

\end{itemize}

\sphinxstepscope


\part{Multivariable Calculus}

\sphinxstepscope


\chapter{Introduction to multi\sphinxhyphen{}variable calculus}
\label{\detokenize{ch/multivariable/intro:introduction-to-multi-variable-calculus}}\label{\detokenize{ch/multivariable/intro:multivariable-calculus}}\label{\detokenize{ch/multivariable/intro::doc}}

\section{Function}
\label{\detokenize{ch/multivariable/intro:function}}\label{\detokenize{ch/multivariable/intro:multivariable-calculus-fun}}

\section{Limit}
\label{\detokenize{ch/multivariable/intro:limit}}\label{\detokenize{ch/multivariable/intro:multivariable-calculus-lim}}

\section{Derivatives}
\label{\detokenize{ch/multivariable/intro:derivatives}}\label{\detokenize{ch/multivariable/intro:multivariable-calculus-der}}

\section{Integrals}
\label{\detokenize{ch/multivariable/intro:integrals}}\label{\detokenize{ch/multivariable/intro:multivariable-calculus-int}}

\section{Theorems}
\label{\detokenize{ch/multivariable/intro:theorems}}

\subsection{Green’s lemma}
\label{\detokenize{ch/multivariable/intro:green-s-lemma}}\label{\detokenize{ch/multivariable/intro:multivariable-calculus-green-lemma}}\begin{equation*}
\begin{split}\begin{aligned}
  \int_{S} \frac{\partial F}{\partial y} dx dy & =     - \oint_{\partial S} F dx \\
  \int_{S} \frac{\partial G}{\partial x} dx dy & = \quad \oint_{\partial S} G dy   
\end{aligned}\end{split}
\end{equation*}\subsubsection*{Proof for simple domains.}

\sphinxAtStartPar
In a simple domain in  \(x\), so that the closed contour \(\partial S\) is delimited by the curves \(y=Y_1(x)\), \(y=Y_2(x) > Y_1(x)\), for \(x \in [x_1, x_2]\),
\begin{equation*}
\begin{split}\begin{aligned}
  \int_{S} \frac{\partial F}{\partial y} dx dy 
  & =   \int_{x=x_1}^{x_2} \int_{y = Y_1(x)}^{Y_2(x)} \frac{\partial F}{\partial y} dy \, dx = \\
  & =   \int_{x=x_1}^{x_2} \left[ F(x,Y_2(x)) - F(x,Y_1(x)) \right] dx = \\
  & = - \int_{x=x_1}^{x_2} F(x,Y_1(x)) - \int_{x=x_2}^{x_1} F(x, Y_2(x)) dx = \\
  & = - \oint_{\partial S} F(x,y) dx 
\end{aligned}\end{split}
\end{equation*}
\sphinxAtStartPar
In a simple domain in  \(y\), so that the closed contour \(\partial S\) is delimited by the curves \(x=X_1(y)\), \(x=X_2(y) > X_1(y)\) for \(y \in [y_1, y_2]\),
\begin{equation*}
\begin{split}\begin{aligned}
  \int_{S} \frac{\partial G}{\partial x} dx dy 
  & = \int_{y=y_1}^{y_2} \int_{x = X_1(y)}^{X_2(y)} \frac{\partial G}{\partial x} dx \, dy = \\
  & = \int_{y=y_1}^{y_2} \left[ G(X_2(y),y) - G(X_1(y),y) \right] dy = \\
  & = \int_{y=y_1}^{y_2} G(X_1(y),y) dy + \int_{y=y_2}^{y_1} G(X_2(y),y) dy = \\
  & = \oint_{\partial S} G(x,y) dy 
\end{aligned}\end{split}
\end{equation*}
\sphinxstepscope


\part{Differential Geometry}

\sphinxstepscope


\chapter{Introduction to Differential Geometry}
\label{\detokenize{ch/differential-geometry/intro:introduction-to-differential-geometry}}\label{\detokenize{ch/differential-geometry/intro:differential-geometry-intro}}\label{\detokenize{ch/differential-geometry/intro::doc}}
\sphinxstepscope


\part{Vector and Tensor Algebra and Calculus}

\sphinxstepscope


\chapter{Tensor Algebra}
\label{\detokenize{ch/tensor-algebra-calculus/algebra:tensor-algebra}}\label{\detokenize{ch/tensor-algebra-calculus/algebra:id1}}\label{\detokenize{ch/tensor-algebra-calculus/algebra::doc}}
\sphinxstepscope


\chapter{Tensor Calculus in Euclidean Spaces}
\label{\detokenize{ch/tensor-algebra-calculus/calculus-euclidean:tensor-calculus-in-euclidean-spaces}}\label{\detokenize{ch/tensor-algebra-calculus/calculus-euclidean:tensor-calculus}}\label{\detokenize{ch/tensor-algebra-calculus/calculus-euclidean::doc}}
\sphinxAtStartPar
This section deals with tensor calculus in Euclidean space or on manifolds embedded in Euclidean spaces, focusing on \(d\)\sphinxhyphen{}dimensional spaces with \(d \le 3\).

\sphinxAtStartPar
This section may rely on results of {\hyperref[\detokenize{ch/differential-geometry/intro:differential-geometry-intro}]{\sphinxcrossref{\DUrole{std,std-ref}{differential geometry}}}}.


\section{Coordinates}
\label{\detokenize{ch/tensor-algebra-calculus/calculus-euclidean:coordinates}}
\sphinxstepscope


\part{Functional Analysis}

\sphinxstepscope


\chapter{Introduction to Functional Analysis}
\label{\detokenize{ch/functional-analysis/intro:introduction-to-functional-analysis}}\label{\detokenize{ch/functional-analysis/intro:functional-analysis}}\label{\detokenize{ch/functional-analysis/intro::doc}}\begin{itemize}
\item {} 
\sphinxAtStartPar
Lebesgue integral

\item {} 
\sphinxAtStartPar
\(L^p\), \(H^p\) function spaces

\item {} 
\sphinxAtStartPar
Banach and Hilbert spaces

\end{itemize}

\sphinxstepscope


\chapter{Distributions (or generalized functions)}
\label{\detokenize{ch/functional-analysis/dirac-delta:distributions-or-generalized-functions}}\label{\detokenize{ch/functional-analysis/dirac-delta:functional-analysis-distributions}}\label{\detokenize{ch/functional-analysis/dirac-delta::doc}}
\sphinxAtStartPar
…


\section{Dirac’s delta}
\label{\detokenize{ch/functional-analysis/dirac-delta:dirac-s-delta}}\label{\detokenize{ch/functional-analysis/dirac-delta:functional-analysis-dirac-delta}}
\sphinxAtStartPar
Dirac’s delta \(\delta(x)\) is a distribution, or generalized function, with the following properties
\begin{enumerate}
\sphinxsetlistlabels{\arabic}{enumi}{enumii}{}{.}%
\item {} 
\end{enumerate}
\begin{equation*}
\begin{split}\int_{D} \delta(x-x_0) \, dx = 1  \quad \text{if $x_0 \in D$}\end{split}
\end{equation*}\begin{enumerate}
\sphinxsetlistlabels{\arabic}{enumi}{enumii}{}{.}%
\setcounter{enumi}{1}
\item {} 
\end{enumerate}
\begin{equation*}
\begin{split}\int_{D} f(x) \delta(x-x_0) \, dx \quad \text{if $x_0 \in D$}\end{split}
\end{equation*}
\sphinxAtStartPar
for \(\forall f(x)\) “regular” \sphinxstylestrong{todo} \sphinxstyleemphasis{what does regular mean?}


\subsection{Dirac’s delta in terms of regular functions}
\label{\detokenize{ch/functional-analysis/dirac-delta:dirac-s-delta-in-terms-of-regular-functions}}
\sphinxAtStartPar
\sphinxstylestrong{Approximations …}
\begin{equation*}
\begin{split}\delta(x) \sim r_{\varepsilon}(x) = \begin{cases} \frac{1}{\varepsilon} & x \in \left[-\frac{\varepsilon}{2}, \frac{\varepsilon}{2} \right] \\ 0 & \text{otherwise} \end{cases}\end{split}
\end{equation*}
\sphinxAtStartPar
as
\begin{enumerate}
\sphinxsetlistlabels{\arabic}{enumi}{enumii}{}{.}%
\item {} 
\sphinxAtStartPar
Unitariety
\begin{equation*}
\begin{split}\int_{x=-\infty}^{\infty} r_{\varepsilon}(x-x_0) \, dx = \int_{x=x_0-\frac{\varepsilon}{2}}^{x_0+\frac{\varepsilon}{2}} \frac{1}{\varepsilon} \, dx = 1 \ ,  \end{split}
\end{equation*}
\sphinxAtStartPar
for \(\forall \varepsilon\);

\item {} 
\sphinxAtStartPar
Shift property, using mean\sphinxhyphen{}value theorem of continuous functions
\begin{equation*}
\begin{split}\int_{x=-\infty}^{\infty} r_{\varepsilon}(x-x_0) f(x) \, dx = \int_{x=x_0-\frac{\varepsilon}{2}}^{x_0+\frac{\varepsilon}{2}} \frac{1}{\varepsilon} f(x) \, dx = \frac{1}{\varepsilon} \varepsilon f(\xi) \ ,  \end{split}
\end{equation*}
\sphinxAtStartPar
with \(\xi \in \left[x_0-\frac{\varepsilon}{2}, x_0+\frac{\varepsilon}{2}\right]\), for the mean value theorem. As \(\varepsilon \rightarrow 0\), \(\xi \rightarrow x_0\), and thus
\begin{equation*}
\begin{split}\int_{x=-\infty}^{\infty} r_{\varepsilon}(x-x_0) f(x) \, dx \rightarrow f(x_0) \end{split}
\end{equation*}
\end{enumerate}
\begin{equation*}
\begin{split}\delta(x) \sim t_{\varepsilon}(x) = \begin{cases} \frac{2}{\varepsilon} \left( 1 - \frac{2 |x|}{\varepsilon} \right) & x \in \left[-\frac{\varepsilon}{2}, \frac{\varepsilon}{2} \right] \\ 0 & \text{otherwise} \end{cases}\end{split}
\end{equation*}
\sphinxAtStartPar
as
\begin{enumerate}
\sphinxsetlistlabels{\arabic}{enumi}{enumii}{}{.}%
\item {} 
\sphinxAtStartPar
Unitariety
\begin{equation*}
\begin{split}\int_{x=-\infty}^{\infty} t_{\varepsilon}(x-x_0) \, dx = \int_{x=x_0-\frac{\varepsilon}{2}}^{x_0+\frac{\varepsilon}{2}} \frac{2}{\varepsilon} \left( 1 - \frac{2 |x|}{\varepsilon} \right) \, dx = \frac{1}{2} \varepsilon \frac{2}{\varepsilon} = 1 \ ,  \end{split}
\end{equation*}
\sphinxAtStartPar
for \(\forall \varepsilon\);

\item {} 
\sphinxAtStartPar
Shift property, using mean\sphinxhyphen{}value integration scheme in \(x \in \left[x_0-\frac{\varepsilon}{2}, x_0 \right]\),  \(x \in \left[x_0, x_0+\frac{\varepsilon}{2} \right]\) (\sphinxstylestrong{todo} \sphinxstyleemphasis{why?})
\begin{equation*}
\begin{split}\begin{aligned}
   \int_{x=-\infty}^{\infty} t_{\varepsilon}(x-x_0) f(x) \, dx
   & = \int_{x=x_0-\frac{\varepsilon}{2}}^{x_0+\frac{\varepsilon}{2}} \frac{2}{\varepsilon} \left( 1 - \frac{2 |x-x_0|}{\varepsilon} \right)  f(x) \, dx = \\
   & = \int_{x=x_0-\frac{\varepsilon}{2}}^{x_0} \frac{2}{\varepsilon} \left( 1 - \frac{2 |x-x_0|}{\varepsilon} \right)  f(x) \, dx 
     + \int_{x=x_0}^{x_0+\frac{\varepsilon}{2}} \frac{2}{\varepsilon} \left( 1 - \frac{2 |x-x_0|}{\varepsilon} \right)  f(x) \, dx = \\
   & = \frac{\varepsilon}{2} \frac{2}{\varepsilon} \left( 1 - \frac{2}{\varepsilon}\frac{\varepsilon}{4} \right)  f\left(x_0-\frac{\varepsilon}{4} \right) \, dx 
     + \frac{\varepsilon}{2} \frac{2}{\varepsilon} \left( 1 - \frac{2}{\varepsilon}\frac{\varepsilon}{4} \right)  f\left(x_0+\frac{\varepsilon}{4} \right) \, dx = \\
   & = \frac{1}{2} f\left( x_0 - \frac{\varepsilon}{4} \right) + \frac{1}{2} f\left( x_0 + \frac{\varepsilon}{4} \right)
   \end{aligned}\end{split}
\end{equation*}
\sphinxAtStartPar
As \(\varepsilon \rightarrow 0\)
\begin{equation*}
\begin{split}\int_{x=-\infty}^{\infty} t_{\varepsilon}(x-x_0) f(x) \, dx \rightarrow f(x_0) \end{split}
\end{equation*}
\end{enumerate}

\sphinxAtStartPar
\sphinxstylestrong{Approximation 1.} For \(\alpha \rightarrow +\infty\),
\begin{equation*}
\begin{split}\varphi_{\alpha}(x) = \sqrt{\frac{\alpha}{\pi}}e^{-\alpha x^2} \sim \delta(x)\end{split}
\end{equation*}
\sphinxAtStartPar
Fourier transform of \(\varphi_{\alpha}(x)\) reads
\begin{equation*}
\begin{split}\begin{aligned}
 \mathscr{F}\{ \varphi_{\alpha}(x) \}(k)
 & = \int_{x=-\infty}^{+\infty} \varphi_\alpha(x) e^{-ikx} \, dx = \\
 & = \int_{x=-\infty}^{+\infty} \sqrt{\frac{\alpha}{\pi}} e^{-\alpha x^2} e^{-ikx} \, dx = \\
 & = \sqrt{\frac{\alpha}{\pi}} \int_{x=-\infty}^{+\infty} e^{-\alpha \left( x + i \frac{k}{2 \alpha} \right)^2} \, dx \, e^{-\frac{k^2}{4 \alpha}} = \\
 & = \sqrt{\frac{\alpha}{\pi}} \, \sqrt{\frac{\pi}{\alpha}} \, e^{-\frac{k^2}{4 \alpha}} =  e^{-\frac{k^2}{4 \alpha}} \ ,\\
\end{aligned}\end{split}
\end{equation*}
\sphinxAtStartPar
for \(\alpha \rightarrow +\infty\),
\begin{equation*}
\begin{split}\mathscr{F}\{ \varphi_{\alpha}(x) \}(k) \rightarrow 1\end{split}
\end{equation*}
\sphinxAtStartPar
and thus \(\varphi_\alpha(x) \rightarrow \delta(x)\) for \(\alpha \rightarrow +\infty\).

\sphinxAtStartPar
\sphinxstylestrong{Approximation 2.} For \(a \rightarrow +\infty\)
\begin{equation*}
\begin{split}\frac{1}{2 \pi} \int_{k=-2\pi a}^{2 \pi a} e^{i k x} \, dk = \int_{y=-a}^{+a} e^{i 2 \pi y x} \, dy \sim \delta(x)\end{split}
\end{equation*}
\sphinxAtStartPar
or
\begin{equation*}
\begin{split}\begin{aligned}
  \delta(x)
  & \sim \frac{1}{2 \pi} \int_{k=-2\pi a}^{2 \pi a} e^{i k x} \, dk = \frac{1}{2 \pi} \left( \int_{k=-2\pi a}^{0} e^{i k x} \, dk +  \int_{0}^{k=2\pi a} e^{i k x} \, dk \right) = \frac{1}{2 \pi} \int_{k = 0}^{2 \pi a} \left( e^{ikx} + e^{ikx} \right) \, dx = \frac{1}{\pi} \int_{x=0}^{2 \pi a} \cos(k x) \, dk \\
  & = \int_{y=-a}^{+a} e^{i 2 \pi y x} \, dy = \dots = \int_{y = 0}^{a} (e^{i 2 \pi y x} + e^{i 2 \pi y x}) \, dy = 2 \int_{y=0}^{a} \cos(2 \pi y x) \, dy \ .
\end{aligned}\end{split}
\end{equation*}
\sphinxAtStartPar
\sphinxstylestrong{Approximation 3.} For \(a \rightarrow +\infty\)
\begin{equation*}
\begin{split}\frac{\sin(2 \pi x a)}{\pi x} \sim \delta(x)\end{split}
\end{equation*}
\sphinxAtStartPar
Directly follows from integral of approximation 2,
\begin{equation*}
\begin{split}\int_{y=-a}^{+a} e^{i 2 \pi y x} \, dy = \frac{1}{i 2 \pi x} \left. e^{i 2 \pi y x}\right|_{y=-a}^{+a} = \frac{1}{\pi x} \frac{e^{i 2 \pi a x} - e^{-i 2 \pi a x}}{2 i} = \frac{\sin(2 \pi x a)}{\pi x}\end{split}
\end{equation*}
\sphinxAtStartPar
\sphinxstylestrong{Approximation 4.} For \(x \in [-\pi, \pi]\), and \(N \rightarrow +\infty\)
\begin{equation*}
\begin{split}\frac{1}{2\pi}\sum_{n=-N}^{N} e^{i n x} = \frac{1}{2 \pi} \frac{\sin\left(\left(N+\frac{1}{2}\right)x\right)}{\sin\left( \frac{x}{2} \right)} \sim \delta(x)\end{split}
\end{equation*}

\subsubsection*{Integral \protect\(I = \int_{-\infty}^{+\infty} e^{-\alpha x^2} \, dx\protect\)}
\begin{equation*}
\begin{split}\begin{aligned}
  I^2 
  & = \int_{x=-\infty}^{+\infty} e^{-\alpha x^2} \, dx \, \int_{y=-\infty}^{+\infty} e^{-\alpha y^2} \, dy = \\
  & = \int_{x=-\infty}^{+\infty} \int_{y=-\infty}^{+\infty} e^{-\alpha (x^2 + y^2)} \, dx \, dy = \\
  & = \int_{\theta=0}^{2\pi} \int_{r=0}^{+\infty} e^{-\alpha r^2} \, r \, dr \, d \theta = \\
  & = 2 \pi \frac{1}{2 \alpha} \int_{r=0}^{+\infty} e^{-\alpha r^2} d \left(\alpha r^2 \right) = \\
  & = \frac{\pi}{\alpha} \left[ - e^{\alpha r^2} \right]\bigg|_{r = 0}^{+\infty} = \frac{\pi}{\alpha} \ .
\end{aligned}\end{split}
\end{equation*}
\sphinxstepscope


\part{Complex Calculus}

\sphinxstepscope


\chapter{Complex Analysis}
\label{\detokenize{ch/complex/analysis:complex-analysis}}\label{\detokenize{ch/complex/analysis:id1}}\label{\detokenize{ch/complex/analysis::doc}}

\section{Complex functions, \protect\(f: \mathbb{C} \rightarrow \mathbb{C}\protect\)}
\label{\detokenize{ch/complex/analysis:complex-functions-f-mathbb-c-rightarrow-mathbb-c}}\label{\detokenize{ch/complex/analysis:complex-analysis-fun}}
\sphinxAtStartPar
A complex function \(f\) of complex variable \(z = x + i y\), \(f: \mathbb{C} \rightarrow \mathbb{C}\), can be written as
\begin{equation*}
\begin{split}f(z) = \tilde{u}(z) + i \tilde{v}(z) = u(x,y) + i v(x,y) \ ,\end{split}
\end{equation*}
\sphinxAtStartPar
as the sum of its real part \(u(z)\) and \(i\) times its imaginary part \(v(x,y)\). Here \(x,y \in \mathbb{R}\), while \(\tilde{u}(z), \tilde{v}(z): \mathbb{C} \rightarrow \mathbb{R}\) and \(u(x,y), v(x,y): \mathbb{R}^2 \rightarrow \mathbb{R}\). With some abuse of notation, tilde won’t be always explicitly written when arguments of real and imaginary parts of \(f\) functions won’t be written.


\subsection{Limit}
\label{\detokenize{ch/complex/analysis:limit}}\label{\detokenize{ch/complex/analysis:complex-analysis-fun-limit}}\begin{equation*}
\begin{split}\lim_{z \rightarrow z_0} f(z) = f(z_0) \qquad , \qquad \forall \varepsilon > 0 \ \exists \delta > 0 \ \text{ s.t. } \  |f(z) - f(z_0)| < \delta \ \forall z \text{ s.t. } |z - z_0| < \varepsilon, \ z \ne z_0 \ .\end{split}
\end{equation*}

\subsection{Derivative}
\label{\detokenize{ch/complex/analysis:derivative}}\label{\detokenize{ch/complex/analysis:complex-analysis-fun-derivative}}
\sphinxAtStartPar
Using the definition of {\hyperref[\detokenize{ch/complex/analysis:complex-analysis-fun-derivative}]{\sphinxcrossref{\DUrole{std,std-ref}{limit of complex functions}}}}, the derivative of a function \(f: \mathbb{C} \rightarrow \mathbb{C}\), if it exists, is the limit of incremental ratio,
\begin{equation*}
\begin{split}f'(z) = \lim_{\Delta z \rightarrow 0} \frac{f(z + \Delta z) - f(z)}{\Delta z} \ .\end{split}
\end{equation*}

\subsection{Line Integrals}
\label{\detokenize{ch/complex/analysis:line-integrals}}\label{\detokenize{ch/complex/analysis:complex-analysis-fun-line-integral}}
\sphinxAtStartPar
Given a line \(\gamma \in \mathbb{C}\), whose parametric form is \(z(s)\), with regular parametrization with parameter \(s \in [s_0, s_1]\),
\begin{equation*}
\begin{split}\int_{\gamma} f(z) \, dz = \int_{s=s_0}^{s_1} f(z(s)) \, z'(s) \, ds \ .\end{split}
\end{equation*}

\section{Holomorphic Functions \sphinxhyphen{} Analytic Functions}
\label{\detokenize{ch/complex/analysis:holomorphic-functions-analytic-functions}}\label{\detokenize{ch/complex/analysis:complex-analysis-holo-fun}}\label{ch/complex/analysis:definition-0}
\begin{sphinxadmonition}{note}{Definition 1}



\sphinxAtStartPar
A holomorphic function is a function whose {\hyperref[\detokenize{ch/complex/analysis:complex-analysis-fun-derivative}]{\sphinxcrossref{\DUrole{std,std-ref}{derivative}}}} exists.
\end{sphinxadmonition}

\sphinxAtStartPar
\sphinxstylestrong{Examples of analytic functions.} \sphinxstylestrong{todo}…


\subsection{Cauchy\sphinxhyphen{}Riemann conditions}
\label{\detokenize{ch/complex/analysis:cauchy-riemann-conditions}}\label{\detokenize{ch/complex/analysis:complex-analysis-holo-fun-cauchy-riemann}}
\sphinxAtStartPar
For a holomorphic function \(f(z) = u(x,y) + i v(x,y)\), Cauchy\sphinxhyphen{}Riemann conditions
\begin{equation*}
\begin{split}\begin{cases}
u_{/x} = v_{/y} \\
u_{/y} = - v_{/x}
\end{cases}\end{split}
\end{equation*}
\sphinxAtStartPar
hold. The evaluation of the derivative once with \(\Delta z = \Delta x\) and once with \(\Delta z = i \Delta y\)
\begin{equation*}
\begin{split}\begin{aligned}
& f'(z) = \lim_{\Delta z \rightarrow 0} \frac{f(z+\Delta z) - f(z)}{\Delta z} = \\ 
& = \left\{
\begin{aligned}
  \lim_{\Delta x \rightarrow 0} \frac{f(x+\Delta x,y) - f(x,y)}{\Delta x} = \lim_{\Delta x \rightarrow 0} \frac{u(x+\Delta x,y) + i v(x+\Delta x,y) - u(x,y) - i v(x,y)}{\Delta x} = u_{/x} + i v_{/x} \\ 
  \lim_{\Delta y \rightarrow 0} \frac{f(x,y+\Delta y) - f(x,y)}{i \Delta y} = \lim_{\Delta y \rightarrow 0}  \frac{u(x,y+\Delta y) + i v(x,y+\Delta y) - u(x,y) - i v(x,y)}{i \Delta y} = -i u_{/y} + v_{/y} \\ 
\end{aligned}
\right.
\end{aligned}\end{split}
\end{equation*}
\sphinxAtStartPar
provides the proof.


\subsection{Cauchy Theorem}
\label{\detokenize{ch/complex/analysis:cauchy-theorem}}\label{\detokenize{ch/complex/analysis:complex-analysis-holo-fun-cauchy-thm}}
\sphinxAtStartPar
For a holomorphic function \(f\), \(f: \Omega \subseteq \mathbb{C} \rightarrow \mathbb{C}\)
\begin{equation*}
\begin{split}\oint_{\gamma} f(z) \, dz = 0 \ ,\end{split}
\end{equation*}
\sphinxAtStartPar
for \(\forall \gamma \subset \Omega\). Proof follows from {\hyperref[\detokenize{ch/multivariable/intro:multivariable-calculus-green-lemma}]{\sphinxcrossref{\DUrole{std,std-ref}{Green’s lemma}}}}, and {\hyperref[\detokenize{ch/complex/analysis:complex-analysis-holo-fun-cauchy-riemann}]{\sphinxcrossref{\DUrole{std,std-ref}{Cauchy\sphinxhyphen{}Riemann conditions}}}}
\begin{equation*}
\begin{split}\begin{aligned}
  \oint_{\gamma} f(z) dz & = \oint_{\gamma} \left( u(x,y) + i v(x,y) \right) \left( dx + i dy \right) = \\
  & = \oint_{\gamma} \left( u dx - v dy \right) + i \oint_{\gamma} \left( u dy + v dx \right) = \\
  & = - \int_{S} \left( \underbrace{u_{/y} + v_{/x}}_{=0} \right) \, dx \, dy + i \int_{S} \left( \underbrace{u_{/x} - v_{/y}}_{=0}  \right) \, dx \, dy = 0 \ .
\end{aligned}\end{split}
\end{equation*}

\section{Useful integrals}
\label{\detokenize{ch/complex/analysis:useful-integrals}}\label{\detokenize{ch/complex/analysis:complex-analysis-useful-int}}

\subsection{Independence of line integral for holomorphic functions}
\label{\detokenize{ch/complex/analysis:independence-of-line-integral-for-holomorphic-functions}}\label{\detokenize{ch/complex/analysis:complex-analysis-useful-int-path-independence}}
\sphinxAtStartPar
For a function \(f(z)\) analytic in \(D\), the line integral on paths \(\ell_{ab,i}\) with the same extreme points \(a\), \(b\) contained in \(D\) is independent on the path, but only depends on the extreme points \(a\), \(b\),
\begin{equation*}
\begin{split}\int_{\ell_{ab,1}} f(z) \, dz = \int_{\ell_{ab,2}} f(z) \, dz\end{split}
\end{equation*}
\sphinxAtStartPar
The proof readily follows, using {\hyperref[\detokenize{ch/complex/analysis:complex-analysis-holo-fun-cauchy-thm}]{\sphinxcrossref{\DUrole{std,std-ref}{Cauchy theorem}}}} applied to a function \(f(z): D \subseteq \mathbb{C} \rightarrow \mathbb{C}\), analytic in \(D\), and splitting the closed path \(\gamma\) into two paths \(\ell_1\), \(\ell_2\) with the same extreme points, \(\gamma = \ell_1 \cup (- \ell_2)\)
\begin{equation*}
\begin{split}0 = \oint_{\gamma} f(z) \, dz = \int_{\ell_1} f(z) \, dz + \int_{-\ell_2} f(z) \, dz = \int_{\ell_1} f(z) \, dz - \int_{\ell_2} f(z) \, dz \ .\end{split}
\end{equation*}

\subsection{Sum and difference of line integrals}
\label{\detokenize{ch/complex/analysis:sum-and-difference-of-line-integrals}}\label{\detokenize{ch/complex/analysis:complex-analysis-useful-int-path-independence-sum}}

\subsection{Integral of \protect\(z^n\protect\)}
\label{\detokenize{ch/complex/analysis:integral-of-z-n}}\label{\detokenize{ch/complex/analysis:complex-analysis-useful-int-path-independence-z-n}}
\sphinxAtStartPar
Given a path \(\gamma\) embracing \(z=0\) only once in counter\sphinxhyphen{}clockwise direction, and \(n \in \mathbb{Z}\)
\begin{equation*}
\begin{split}\oint_{\gamma} z^n \, dz = \left\{ \begin{aligned}  2 \pi i & \qquad \text{if $n = -1$} \\ 0 & \qquad \text{otherwise} \end{aligned} \right.\end{split}
\end{equation*}
\sphinxAtStartPar
Since \(z^n\) is analytic everywhere (\sphinxstylestrong{todo} \sphinxstyleemphasis{prove it! Add a section with proofs for common functions}) except for \(z=0\), it’s possible to evaluate the integral on a circle with center \(z=0\) and radius \(R\). Using polar expression of the complex numbers on the circle, \(z = R e^{i \theta}\), \(\theta \in [0, 2 \pi]\), \(R\) const, the differential becomes \(dz = i R e^{i \theta} d \theta\) and the integral
\begin{equation*}
\begin{split}\begin{aligned}
\oint_{\gamma} z^n \, dz
  & = \int_{\theta=0}^{2 \pi} \left( R e^{i\theta}\right)^n i R e^{i \theta} d \theta = \\
  & = i \int_{\theta=0}^{2 \pi} R^{n+1} e^{i (n+1) \theta} d \theta = \\
  & = \left\{ \begin{aligned}
    & \text{if $n=-1$} & : & \quad  i 2 \pi \\
    & \text{otherwise} & : & \quad  i R^{n+1} \frac{1}{i(n+1)} \left.e^{i(n+1)\theta}\right|_{\theta=0}^{2\pi} = \frac{R^{n+1}}{n+1} ( 1 - 1 ) = 0 \\
  \end{aligned} \right.\\
\end{aligned}\end{split}
\end{equation*}

\section{Meromorphic functions}
\label{\detokenize{ch/complex/analysis:meromorphic-functions}}\label{\detokenize{ch/complex/analysis:complex-analysis-mero-fun}}\label{ch/complex/analysis:definition-1}
\begin{sphinxadmonition}{note}{Definition 2}



\sphinxAtStartPar
A meromorphic function in a domain is a function holomorphic everywhere except for a (finite?) number of poles. \sphinxstylestrong{check}
\end{sphinxadmonition}


\subsection{Singularities}
\label{\detokenize{ch/complex/analysis:singularities}}\label{\detokenize{ch/complex/analysis:complex-analysis-singularities}}\label{ch/complex/analysis:definition-2}
\begin{sphinxadmonition}{note}{Definition 3 (Pole)}



\sphinxAtStartPar
A pole of order \(n\) of a function \(f(z)\) is a complex number \(a\) so that
\begin{equation*}
\begin{split}f(z) = \frac{\phi(z)}{(z-a)^n} \ ,\end{split}
\end{equation*}
\sphinxAtStartPar
with \(\phi(z)\) holomorphic in \(\phi(a) \ne 0\)
\end{sphinxadmonition}

\sphinxAtStartPar
\sphinxstylestrong{Examples.} …
\label{ch/complex/analysis:definition-3}
\begin{sphinxadmonition}{note}{Definition 4 (Branch)}


\end{sphinxadmonition}

\sphinxAtStartPar
\sphinxstylestrong{Examples.} \(f(z) = z^{\frac{1}{2}}\)
\label{ch/complex/analysis:definition-4}
\begin{sphinxadmonition}{note}{Definition 5 (Removable singularities)}


\end{sphinxadmonition}

\sphinxAtStartPar
\sphinxstylestrong{Example.} \(f(z) = \frac{\sin z}{z}\)

\sphinxAtStartPar
\sphinxstylestrong{Other irregularities.}


\subsection{Laurent Series}
\label{\detokenize{ch/complex/analysis:laurent-series}}\label{\detokenize{ch/complex/analysis:complex-analysis-mero-fun-laurent}}
\sphinxAtStartPar
Given a function \(f(z)\), in a disk \(D_{a,\varepsilon}: 0 < |z-a| < \varepsilon\), its Laurent series centered in \(a\) is the convergent (to \(f(z)\), \sphinxstylestrong{todo} \sphinxstyleemphasis{which type of convergnence?}) series
\begin{equation}\label{equation:ch/complex/analysis:eq:laurent}
\begin{split}f(z) \sim \sum_{n=-\infty}^{+\infty} a_n (z-a)^n \ ,\end{split}
\end{equation}
\sphinxAtStartPar
with
\begin{equation}\label{equation:ch/complex/analysis:eq:laurent:coeff}
\begin{split}a_n = \frac{1}{2 \pi i}\int_{\gamma} f(z) \, (z-a)^{-(n+1)} \, dz\end{split}
\end{equation}
\sphinxAtStartPar
and \(\gamma\) embracing \(z = a\) once counter\sphinxhyphen{}clockwise. Proof follows immediately inserting the expressions of the coefficients \(a_n\) and using the {\hyperref[\detokenize{ch/complex/analysis:complex-analysis-useful-int-path-independence-z-n}]{\sphinxcrossref{\DUrole{std,std-ref}{integral of \(z^n\)}}}}. Evaluating the integral \eqref{equation:ch/complex/analysis:eq:laurent:coeff} of the coefficients of the Laurent series, using \eqref{equation:ch/complex/analysis:eq:laurent} to replace \(f(z)\) with its series
\begin{equation*}
\begin{split}\begin{aligned}
  a_n & = \frac{1}{2 \pi i}\oint_{\gamma} \sum_{m=-\infty}^{+\infty} a_m (z-a)^m (z-a)^{-(n+1)} = \\
  & = \frac{1}{2 \pi i} \oint_{\gamma} \sum_{m=-\infty}^{+\infty} a_m (z-a)^{m - n - 1}  \, dz = \\
  & = \frac{1}{2 \pi i} \oint_{\gamma} a_n \, z^{-1} \, dz = \\
  & = a_n \ . 
\end{aligned}\end{split}
\end{equation*}
\sphinxAtStartPar
\sphinxstylestrong{todo} \sphinxstyleemphasis{Some freestyle with function and its convergent series…add some detail, and the meaning of convergence}


\subsection{Cauchy formula}
\label{\detokenize{ch/complex/analysis:cauchy-formula}}\label{\detokenize{ch/complex/analysis:complex-analysis-mero-fun-cauchy-formula}}
\sphinxAtStartPar
For an analytic function \(f(z)\),
\begin{equation*}
\begin{split}f(a) = \frac{1}{2 \pi i} \oint_{\gamma} \frac{f(z)}{z-a} \, dz\end{split}
\end{equation*}
\sphinxAtStartPar
Proof readily follows using the {\hyperref[\detokenize{ch/complex/analysis:complex-analysis-useful-int-path-independence-z-n}]{\sphinxcrossref{\DUrole{std,std-ref}{integral of \(z^n\)}}}} on the Taylor series of \(\frac{f(z)}{z-a}\) whose \(0^{th}\) order term reads \(f(a)\),
\begin{equation*}
\begin{split}\frac{1}{2\pi i} \oint_{\gamma} \frac{f(a)+\sum_{m=1}^{+\infty} f'(a) (z-a)^m}{z-a} \, dz = \frac{1}{2\pi i} \oint_{\gamma} \frac{f(a)}{z-a} \, dz = f(a) \frac{2 \pi i}{2 \pi i} = f(a) \ .\end{split}
\end{equation*}

\subsection{Residues}
\label{\detokenize{ch/complex/analysis:residues}}\label{\detokenize{ch/complex/analysis:complex-analysis-mero-fun-residues}}\label{ch/complex/analysis:definition-5}
\begin{sphinxadmonition}{note}{Definition 6 (Residue)}



\sphinxAtStartPar
The residue of function \(f\) in \(a\), \(\text{Res}(f,a)\) is a complex number \(R\) so that \(f(z) - \frac{R}{(z-a)}\) has analytic antiderivative in a disk \(D_{a,\varepsilon}: \ 0 < |z-a| < \varepsilon\).
\end{sphinxadmonition}

\sphinxAtStartPar
\sphinxstylestrong{todo} Explain this definition. Couldn’t be possible to use \(\text{Res}(f,a) = \frac{1}{2 \pi i} \oint_{\gamma} f(z) \, dz = a_{-1}\) instead?

\sphinxAtStartPar
\sphinxstylestrong{Properties.}
\begin{itemize}
\item {} 
\sphinxAtStartPar
If \(f(z)\) is analytic in \(D_{a,\varepsilon}\) and has a pole of order \(n\) in \(z = a\), its Laurent series has \(a_m=0\) for \(m < n\) and reads
\begin{equation}\label{equation:ch/complex/analysis:eq:laurent:pole-n}
\begin{split}f(z) = \sum_{m=-n}^{+\infty} a_m (z-a)^m \ ,\end{split}
\end{equation}
\sphinxAtStartPar
with \(a_{-n} \ne 0\). Since \(f(z)\) has a pole of order \(n\) in \(z = a\), it can be written as
\begin{equation*}
\begin{split}f(z) = \frac{\phi(z)}{(z-a)^n} \ ,\end{split}
\end{equation*}
\sphinxAtStartPar
with \(\phi(z)\) analytic in \(D_{a,\varepsilon}\) and \(\phi(a) \ne 0\). Since \(\phi(z)\) is analytic, it has a Taylor series (or a Laurent series with non\sphinxhyphen{}negative powers),
\begin{equation*}
\begin{split}\phi(z) \sim \sum_{m=0}^{+\infty} b_m (z-a)^m \ ,\end{split}
\end{equation*}
\sphinxAtStartPar
(\sphinxstylestrong{todo} \sphinxstyleemphasis{prove it! Extension of the real case. Add a link to the proof}) and thus
\begin{equation*}
\begin{split}f(z) \sim \sum_{m=0}^{+\infty} b_m (z-a)^{m-n} = \sum_{m=-n}^{+\infty} b_{m+n} (z-a)^{m} = \sum_{m=-n}^{+\infty} a_{m} (z-a)^m \ , \end{split}
\end{equation*}
\sphinxAtStartPar
with \(a_m = b_{m+n}\).

\item {} 
\sphinxAtStartPar
For simple closed path \(\gamma\) (embracing \(a\) only once counter\sphinxhyphen{}clokwise) in \(D_{a, \varepsilon}\),
\begin{equation}\label{equation:ch/complex/analysis:eq:residue-thm:0}
\begin{split}\oint_{\gamma} f(z) \, dz = 2 \pi i a_{-1} = 2 \pi i \text{Res}(f,a)\end{split}
\end{equation}
\sphinxAtStartPar
The proof readily follows, using the {\hyperref[\detokenize{ch/complex/analysis:complex-analysis-useful-int-path-independence-z-n}]{\sphinxcrossref{\DUrole{std,std-ref}{integral of \(z^n\)}}}} and Laurent series \eqref{equation:ch/complex/analysis:eq:laurent} of \(f(z)\),
\begin{equation*}
\begin{split}\oint_{\gamma} f(z) \, dz = \oint_{\gamma} \sum_{m=-\infty}^{+\infty} a_m (z-a)^m \, dz = 2 \pi i a_{-1} \ .\end{split}
\end{equation*}
\item {} 
\sphinxAtStartPar
For a pole \(a\) of order \(n\), the following holds
\begin{equation*}
\begin{split}a_{-1} =  \frac{1}{(n+1)!} \lim_{z \rightarrow a} \frac{d^{n-1}}{dz^{n-1}} \left[ (z-a)^n \, f(z) \right]\end{split}
\end{equation*}
\sphinxAtStartPar
The proof follows using Laurent series \{eq\}`eq:laurent:pole\sphinxhyphen{}n\} for a function with pole of order \(n\), and evaluating the \((n-1)^{th}\) order derivative
\begin{equation*}
\begin{split}\begin{aligned}
   \frac{d^{n-1}}{dz^{n-1}} \left[ (z-a)^n f(z) \right] 
    & = \frac{d^{n-1}}{dz^{n-1}} \left[ (z-a)^n \sum_{m=-n}^{+\infty} a_n (z-a)^m \right] = \\
    & = \dfrac{d^{n-1}}{dz^{n-1}} \left[ \sum_{m=-n}^{+\infty} a_n (z-a)^{m+n} \right] = \\
    & = \dfrac{d^{n-1}}{dz^{n-1}} \left[ \sum_{m=0}^{+\infty} a_{m-n} (z-a)^{m} \right] = \\
    & = \dfrac{d^{n-2}}{dz^{n-2}} \left[ \sum_{m=0}^{+\infty} m a_{m-n} (z-a)^{m-1} \right] = \\
    & = \dfrac{d^{n-3}}{dz^{n-3}} \left[ \sum_{m=0}^{+\infty} m(m-1) a_{m-n} (z-a)^{m-2} \right] = \\
    & = \dots = \\
    & = \left[ \sum_{m=0}^{+\infty} m! \, a_{m-n} (z-a)^{m-n+1} \right] \\
  \end{aligned}\end{split}
\end{equation*}
\sphinxAtStartPar
and then letting \(z \rightarrow a\), so that only the term with \(m-n+1 = 0\) survives
\begin{equation*}
\begin{split}\lim_{z \rightarrow a} \frac{d^{n-1}}{dz^{n-1}} \left[ (z-a)^n \sum_{m=-n}^{+\infty} a_n (z-a)^m \right] = (n-1)! \, a_{-1} \ .\end{split}
\end{equation*}
\end{itemize}


\subsection{Residue Theorem}
\label{\detokenize{ch/complex/analysis:residue-theorem}}\label{\detokenize{ch/complex/analysis:complex-analysis-mero-fun-residues-thm}}\label{ch/complex/analysis:theorem-6}
\begin{sphinxadmonition}{note}{Theorem 1 (Residue Theorem)}



\sphinxAtStartPar
Given \(f(z)\) with a finite number of poles \(p_n \in D\), then
\begin{equation*}
\begin{split}\int_{\gamma} f(z) \, dz = 2 \pi i \ \sum_{n} I(\gamma, p_n) \text{Res}(f,p_n) \ ,\end{split}
\end{equation*}
\sphinxAtStartPar
being \(\gamma\) a path in \(D\), and \(I(\gamma, p_n)\) the winding index of the path \(\gamma\) around pole \(p_n\) (+1 for each counter\sphinxhyphen{}clockwise loop, \sphinxhyphen{}1 for each clockwise loop).
\end{sphinxadmonition}

\sphinxAtStartPar
The proof readily follows extending the result for a single pole \eqref{equation:ch/complex/analysis:eq:residue-thm:0} to general number of poles and general paths \(\gamma\) embracing (with sign) each pole \(p_n\) \(I(\gamma,p_n)\) times, with the same techinques shown in section {\hyperref[\detokenize{ch/complex/analysis:complex-analysis-useful-int-path-independence-sum}]{\sphinxcrossref{\DUrole{std,std-ref}{Sum and difference of line integrals}}}}.


\subsection{Evaluation of integrals}
\label{\detokenize{ch/complex/analysis:evaluation-of-integrals}}

\subsection{Inverse Laplace Transform}
\label{\detokenize{ch/complex/analysis:inverse-laplace-transform}}
\sphinxAtStartPar
Given Laplace transform
\begin{equation*}
\begin{split}F(s) := \mathscr{L}\{f(t)\}(s) := \int_{t=0^-}^{+\infty} f(t) e^{-st} \, dt \ ,\end{split}
\end{equation*}
\sphinxAtStartPar
the inverse transform can be evaluated as
\begin{equation*}
\begin{split}f(t) = \mathscr{L}^{-1}\{F(s)\}(t) := \lim_{T \rightarrow +\infty} \frac{1}{2 \pi i} \int_{s = a-iT}^{a+iT} e^{st} F(s) \, ds \ ,\end{split}
\end{equation*}
\sphinxAtStartPar
with \(a > \text{Re}\{p_n\}\) for each pole of the function \(F(s)\), evaluated on the vertical line \(s = a+iy\), \(y \in [-T,T]\), \(ds = i d y\),
\begin{equation*}
\begin{split}\begin{aligned}
  \lim_{T \rightarrow +\infty} \frac{1}{2 \pi i} \int_{s = a-iT}^{a+iT} e^{st} F(s) \, ds 
  & = \lim_{T \rightarrow +\infty} \frac{1}{2 \pi i} \int_{s = a-iT}^{a+iT} e^{st} \int_{\tau=0^-}^{+\infty} f(\tau) e^{-s\tau} \, d \tau  \, ds = \\
  & = \lim_{T \rightarrow +\infty} \frac{1}{2 \pi i} \int_{y = -iT}^{iT} e^{(a+iy)t} \int_{\tau=0^-}^{+\infty} f(\tau) e^{-(a+iy)\tau} \, d \tau  \, i dy = \\
  & = \lim_{T \rightarrow +\infty} \frac{1}{2 \pi} \int_{y = -iT}^{iT} \int_{\tau=0^-}^{+\infty} e^{iy(t-\tau)} e^{a(t-\tau)} f(\tau) \, d \tau  \, dy = \\
  & = \dots \\
  & = \int_{\tau=0^-}^{+\infty} \delta(t-\tau) e^{a(t-\tau)} f(\tau) d \tau = f(t) \ .
\end{aligned}\end{split}
\end{equation*}
\sphinxAtStartPar
having used the transform of {\hyperref[\detokenize{ch/functional-analysis/dirac-delta:functional-analysis-dirac-delta}]{\sphinxcrossref{\DUrole{std,std-ref}{Dirac’s delta}}}} \(\delta(t) = \frac{1}{2\pi} \int_{\omega=-\infty}^{+\infty} e^{-j \omega t} \, d\omega\).

\sphinxAtStartPar
\sphinxstylestrong{todo} \sphinxstyleemphasis{If \(a > \text{Re}\{p_n\}\), the contour build with the vertical line with real part \(a\) and the arc of circumference on its}

\sphinxstepscope


\chapter{Laplace Transform}
\label{\detokenize{ch/complex/laplace:laplace-transform}}\label{\detokenize{ch/complex/laplace:complex-laplace}}\label{\detokenize{ch/complex/laplace::doc}}\begin{equation*}
\begin{split}\mathscr{L}\left\{ f(t) \right\}(s) := \int_{t=0^-}^{+\infty} f(t) \, e^{st} \, dt = F(s) \ .\end{split}
\end{equation*}

\section{Inverse transform}
\label{\detokenize{ch/complex/laplace:inverse-transform}}\begin{equation*}
\begin{split}f(t) = \mathscr{L}^{-1}\left\{ F(s) \right\} = \dots\end{split}
\end{equation*}

\section{Properties}
\label{\detokenize{ch/complex/laplace:properties}}
\sphinxAtStartPar
\sphinxstylestrong{Linearity.}

\sphinxAtStartPar
\sphinxstylestrong{{\hyperref[\detokenize{ch/functional-analysis/dirac-delta:functional-analysis-dirac-delta}]{\sphinxcrossref{\DUrole{std,std-ref}{Dirac delta}}}}.}
\begin{equation*}
\begin{split}\mathscr{L}\left\{ \delta(t) \right\} = \int_{t=0^-}^{+\infty} \delta(t) \, e^{st} \, dt = 1 \end{split}
\end{equation*}
\sphinxAtStartPar
\sphinxstylestrong{Time delay.}

\sphinxAtStartPar
\sphinxstylestrong{Derivative.}

\sphinxAtStartPar
\sphinxstylestrong{Integral.}

\sphinxAtStartPar
\sphinxstylestrong{Initial value.}

\sphinxAtStartPar
\sphinxstylestrong{Final value.}

\sphinxstepscope


\chapter{Fourier Transforms}
\label{\detokenize{ch/complex/fourier:fourier-transforms}}\label{\detokenize{ch/complex/fourier:complex-fourier}}\label{\detokenize{ch/complex/fourier::doc}}\begin{itemize}
\item {} 
\sphinxAtStartPar
Fourier series: continuous time, periodic function in time

\item {} 
\sphinxAtStartPar
Fourier transform: continuous time, non\sphinxhyphen{}periodic function in time

\item {} 
\sphinxAtStartPar
Discrete Fourier transform (DFT):

\item {} 
\sphinxAtStartPar
Discrete time Fourier transform (DTFT):

\end{itemize}


\section{Fourier Series}
\label{\detokenize{ch/complex/fourier:fourier-series}}\label{\detokenize{ch/complex/fourier:complex-fourier-fs}}
\sphinxAtStartPar
For a \(T\)\sphinxhyphen{}periodic function,
\begin{equation*}
\begin{split}g(t) \end{split}
\end{equation*}

\section{Fourier Transform}
\label{\detokenize{ch/complex/fourier:fourier-transform}}\label{\detokenize{ch/complex/fourier:complex-fourier-ft}}\begin{equation*}
\begin{split}\mathscr{F}\left\{ g(t) \right\}(f) := \int_{t = -\infty}^{+\infty} g(t) \, e^{-i 2 \pi f t} \, dt .\end{split}
\end{equation*}

\subsection{Properties}
\label{\detokenize{ch/complex/fourier:properties}}
\sphinxAtStartPar
\sphinxstylestrong{Linearity.}

\sphinxAtStartPar
\sphinxstylestrong{{\hyperref[\detokenize{ch/functional-analysis/dirac-delta:functional-analysis-dirac-delta}]{\sphinxcrossref{\DUrole{std,std-ref}{Dirac delta}}}}.}
\begin{equation*}
\begin{split}\mathscr{L}\left\{ \delta(t) \right\} = \int_{t=-\infty}^{+\infty} \delta(t) \, e^{-i 2 \pi f t} \, dt = 1 \end{split}
\end{equation*}
\sphinxAtStartPar
\sphinxstylestrong{Time delay.}

\sphinxAtStartPar
\sphinxstylestrong{Derivative.}

\sphinxAtStartPar
\sphinxstylestrong{Integral.}

\sphinxAtStartPar
\sphinxstylestrong{Initial value.}

\sphinxAtStartPar
\sphinxstylestrong{Final value.}


\subsection{Inverse Fourier Transform}
\label{\detokenize{ch/complex/fourier:inverse-fourier-transform}}\begin{equation*}
\begin{split}\mathscr{F}^{-1}\left\{ G(f) \right\}(t) := \int_{f = -\infty}^{+\infty} G(f) \, e^{i 2 \pi f t} \, df .\end{split}
\end{equation*}
\sphinxAtStartPar
\sphinxstylestrong{Proof using Dirac’s delta expression.}
\begin{equation*}
\begin{split}\begin{aligned}
\mathscr{F}^{-1}\left\{ G(f) \right\}(t) := \int_{f = -\infty}^{+\infty} G(f) \, e^{i 2 \pi f t} \, df 
  & = \int_{f = -\infty}^{+\infty} \int_{\tau=-\infty}^{+\infty} g(\tau) e^{-i 2 \pi f \tau} \, e^{i 2 \pi f t} \, df = \\ 
  & = \int_{f = -\infty}^{+\infty} \int_{\tau=-\infty}^{+\infty} g(\tau) e^{-i 2 \pi f \tau} \, e^{i 2 \pi f t} \, df = \\ 
  & = \int_{f = -\infty}^{+\infty} \int_{\tau=-\infty}^{+\infty} g(\tau) e^{i 2 \pi f (t-\tau)} \, df = \\ 
  & = \int_{\tau=-\infty}^{+\infty} g(\tau) \delta(t-\tau) \, d\tau = g(t) \ . 
\end{aligned}\end{split}
\end{equation*}
\sphinxAtStartPar
\sphinxstylestrong{Proof.}





\sphinxstepscope


\part{Calculus of Variations}

\sphinxstepscope


\chapter{Introduction to Calculus of Variations}
\label{\detokenize{ch/calculus-variations/intro:introduction-to-calculus-of-variations}}\label{\detokenize{ch/calculus-variations/intro:calculus-variations-intro}}\label{\detokenize{ch/calculus-variations/intro::doc}}





\renewcommand{\indexname}{Proof Index}
\begin{sphinxtheindex}
\let\bigletter\sphinxstyleindexlettergroup
\bigletter{definition\sphinxhyphen{}0}
\item\relax\sphinxstyleindexentry{definition\sphinxhyphen{}0}\sphinxstyleindexextra{ch/complex/analysis}\sphinxstyleindexpageref{ch/complex/analysis:\detokenize{definition-0}}
\indexspace
\bigletter{definition\sphinxhyphen{}1}
\item\relax\sphinxstyleindexentry{definition\sphinxhyphen{}1}\sphinxstyleindexextra{ch/complex/analysis}\sphinxstyleindexpageref{ch/complex/analysis:\detokenize{definition-1}}
\indexspace
\bigletter{definition\sphinxhyphen{}2}
\item\relax\sphinxstyleindexentry{definition\sphinxhyphen{}2}\sphinxstyleindexextra{ch/complex/analysis}\sphinxstyleindexpageref{ch/complex/analysis:\detokenize{definition-2}}
\indexspace
\bigletter{definition\sphinxhyphen{}3}
\item\relax\sphinxstyleindexentry{definition\sphinxhyphen{}3}\sphinxstyleindexextra{ch/complex/analysis}\sphinxstyleindexpageref{ch/complex/analysis:\detokenize{definition-3}}
\indexspace
\bigletter{definition\sphinxhyphen{}4}
\item\relax\sphinxstyleindexentry{definition\sphinxhyphen{}4}\sphinxstyleindexextra{ch/complex/analysis}\sphinxstyleindexpageref{ch/complex/analysis:\detokenize{definition-4}}
\indexspace
\bigletter{definition\sphinxhyphen{}5}
\item\relax\sphinxstyleindexentry{definition\sphinxhyphen{}5}\sphinxstyleindexextra{ch/complex/analysis}\sphinxstyleindexpageref{ch/complex/analysis:\detokenize{definition-5}}
\indexspace
\bigletter{theorem\sphinxhyphen{}6}
\item\relax\sphinxstyleindexentry{theorem\sphinxhyphen{}6}\sphinxstyleindexextra{ch/complex/analysis}\sphinxstyleindexpageref{ch/complex/analysis:\detokenize{theorem-6}}
\end{sphinxtheindex}

\renewcommand{\indexname}{Index}
\printindex
\end{document}